% !TEX root = ./root.tex

% Modulo spaces

\renewcommand{\pod}[1]{\allowbreak\mathchoice
  {\if@display \mkern 18mu\else \mkern 8mu\fi (#1)}
  {\if@display \mkern 18mu\else \mkern 8mu\fi (#1)}
  {\mkern4mu(#1)}
  {\mkern4mu(#1)}
}

% Theorems defitions

\newcommand{\autotheorem}[3]{
\newaliascnt{#1counter}{#2}
\newtheorem{#1}[#1counter]{#3}
\expandafter\newcommand\csname #1counterautorefname\endcsname{#3}
}

\theoremstyle{plain}
\newtheorem{base_theorem}{BaseTheorem}

\autotheorem{theorem}{base_theorem}{Theorem}
\autotheorem{lemma}{base_theorem}{Lemma}
\autotheorem{proposition}{base_theorem}{Proposition}
\autotheorem{corollary}{base_theorem}{Corollary}
\autotheorem{fact}{base_theorem}{Fact}
\autotheorem{conjecture}{base_theorem}{Conjecture}


\theoremstyle{definition}
\autotheorem{definition}{base_theorem}{Definition}
\autotheorem{example}{base_theorem}{Example}

\theoremstyle{remark}
\autotheorem{remark}{base_theorem}{Remark}

\newtheoremstyle{case}%
  {\topsep}%
  {\topsep}%
  {}%
  {\parindent}%
  {\itshape}%
  {}%
  { }%
  {\thmname{#1}\thmnumber{ #2}:{\thmnote{ #3.}}}

\theoremstyle{case}
\newtheorem{case}{Case}
\counterwithin*{case}{base_theorem}
\providecommand*{\caseautorefname}{Case}
